\section{Experiments}

\subsection{\emph{E.~Coli}}

We used the reference genome for the \emph{E.~Coli} bacterium available in \url{http://ftp.ensemblgenomes.org/pub/bacteria/release-52/fasta/bacteria_0_collection/escherichia_coli_str_k_12_substr_mg1655_gca_000005845/dna/}

\subsubsection{Synthetic reads without errors}

As a first experiment, we generated synthetic reads without errors by reading substrings of length \textit{250} from the reference genome starting at
a random position. This was repeated a number of times in order to give the desired coverage of \textit{80x} (i.e.: $\frac{|\mathit{genome}| \times 80}{250}$).

\subsubsection{Synthetic reads with errors}

In order to simulate the reads as they would be produced by the sequencing process, we used the ART Illumina toolkit to generate
synthetic reads from the \emph{E.~Coli} genome. The reads were generated using the following parameters:

\begin{enumerate}
\item \textbf{Sequencing System}: Illumina MiSeq v3
\item \textbf{Read length}: \textit{250bp}
\item \textbf{Coverage}: \textit{80x}
\end{enumerate}