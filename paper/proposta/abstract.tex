\section{Resumo}

O uso de memória causa um dos maiores \emph{bottlenecks} nos \emph{softwares} de reconstrução
de genomas (\emph{assemblers}). Um dos principais alvos de otimização nesse tipo de programa
é a representação de um Grafo de \emph{de Bruijn} (GdB), uma vez que essa estrutura é amplamente utilizada
por esses \emph{assemblers}, apesar de requerer um alto custo em memória (na ordem de GBs).
Este trabalho se propõe a analisar soluções atuais para esse problema, e propor um novo modelo
de representação de GdBs eficiente em memória. Além disso, será dado um foco em estruturas de dados
baseadas em \emph{sketches}, permitindo que as leituras não precisem estar todas disponíveis simultaneamente,
reduzindo, também, os requisitos de armazenamento e, potencialmente, tempo necessário para a reconstrução
do genoma.