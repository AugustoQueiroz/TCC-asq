\section{Contexto}

As tecnologias de sequenciamento de genomas de segunda geração revolucionaram
o acesso a pesquisa genomica, reduzindo os custos e esforços necessários para
esse tipo de estudo\cite{Imelfort2009}. À medida que esse
tipo de pesquisa se tornou mais acessível, surgiu, também, a importância de 
desenvolvimento de métodos de reconstrução de sequências de genoma, partindo
dos dados gerados pelos sistemas de gerenciamento, que fossem mais eficientes.
A montagem de sequências é um problema \emph{NP-difícil}
\cite{Medvev2007}, de forma que tentativas de otimização do processo de reconstrução
não são muito exploradas. Assim, muitos dos esforços de otimização almejaram, então, a
redução no uso de memória, em particular para a representação de Grafos de \emph{de Bruijn} (GdBs), 
estrutura que ganhou grande relevância nos \emph{assemblers} baseados nas novas
tecnologias de sequenciamento\cite{Conway2011}\cite{Chikhi2014} graças ao fato destas
produzirem uma enorme quantidade de leituras (na ordem de milhões de leituras)
curtas (normalmente entre 25 e 400bp\footnote{bp, ou \emph{base pairs}, são os
pares de base $\{A, C, G, T\}$ que compõem uma sequência de DNA}).

Diversas representações para os GdBs já foram propostas\cite{Chikhi2014} visando, principalmente,
obter uma melhor eficiência em memória enquanto garantem uma maior exatidão da informação
(pois algumas representações aceitam alguma taxa de falsos positivos em troca de
um melhor aproveitamento do espaço, como é o caso da representação introduzida em
\cite{Pell13272}).

Uma perspectiva ainda pouco explorada é a de representações dos GdBs em configuração
de \emph{data stream}, na qual os dados não são todos disponibilizados simultânemanete,
sendo adicionados na estrutura a medida que as leituras são realizadas. O único
trabalho encontrado nesse sentido foi o \emph{assembler} \emph{FastEtch}\cite{Ghosh2016},
que faz uso de um \emph{sketch} \emph{CountMin}. Esse tipo de
exploração tem potencial para impactar tanto o requerimento de memória do sistema, quanto
o desempenho em tempo necessário para a montagem do genoma, uma vez que o sistema poderia
montar o GdB a medida que as leituras são realizadas, não precisando que todas
elas sejam armazenadas em disco e disponibilizadas simultâneamente.